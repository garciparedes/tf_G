% !TEX root = ../../document.tex

\documentclass{subfiles}

\begin{document}

  \chapter{Algoritmo PageRank}
  \label{chap:graphs}

    \section{Introducción}
    \label{sec:pagerank_intro}

      \paragraph{}
      [TODO]

    \section{Caminos Aleatorios}
    \label{sec:random_walks}

      \paragraph{}
      [TODO ]

      \subsection{Cadenas de Markov}
      \label{sec:markov_chains}

        \paragraph{}
        [TODO ]

      \subsection{Matriz Laplaciana}
      \label{sec:laplacian_matrix}

        \paragraph{}
        La \emph{matriz laplaciana} consiste en una estrategia de representación de grafos que cumple importantes propiedades a partir de las cuales se facilita en gran medida la resolución de un gran número de problemas sobre grafos. [TODO ]

      \subsection{Matriz Laplaciana de Caminos Aleatorios Normalizada}
      \label{sec:random_walk_normalized_laplacian_matrix}

        \paragraph{}
        [TODO ]

    \section{Definición Formal}
    \label{sec:pagerank_formal_definition}

      \paragraph{}
      [TODO]

    \section{Algoritmo Básico}
    \label{sec:pagerank_algorithm}

      \paragraph{}
      [TODO ]

    \section{Soluciones Aproximadas}
    \label{sec:pagerank_algorithm_approximated}

      \paragraph{}
      [TODO ]

    \section{Conclusiones}
    \label{sec:pagerank_conclusions}

      \paragraph{}
      [TODO]

\end{document}
