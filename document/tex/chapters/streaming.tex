% !TEX root = ../../document.tex

\documentclass{subfiles}

\begin{document}

  \chapter{Algoritmos para Streaming}
  \label{chapter:streaming}


    \section{Introducción}
    \label{sec:streaming_intro}

      \paragraph{}
      En este capítulo se trata de realizar una descripción en profundidad acerca de los \emph{Algoritmos en Streaming} desde una perspectiva tanto a teórica como práctica. Para ello se describirá el modelo de cómputo en que se enmarcan dichos algoritmos (Modelo en Streaming) en la sección \ref{sec:streaming_model} además de su estructura básica en la sección \ref{sec:streaming_structure}. El motivo de dicha descripción se debe a que los \emph{Algoritmos para Streaming} presentan un conjunto de peculiaridades respecto de la gran mayoría de algoritmos utilizados comunmente. [TODO introducir el resto de secciones]

      \paragraph{}
      Para realizar una primera aproximación acerca de en qué consiste esta categoría algorítmica es necesario realizar una diferenciación entre distintos conceptos relacionados con ella, que pueden producir confusiones debido a su similitud o abusos previos del lenguaje. Por lo tanto, a continuación se describen conceptos relacionados con los \emph{Algoritmos en Streaming} que permitirán introducir al lector en el contexto del problema. Además, se realiza una diferenciación acerca de los factores que se pretenden optimizar a partir de esta estrategia de diseño de algoritmos.


      \subsection{Computación en Tiempo Real}
      \label{sec:realtime_computing}
        \paragraph{}
        El primer concepto que se describe es \textbf{Computación en Tiempo Real}, que tal y cómo describen Shin y Ramanathan \cite{259423} se carácteriza por tres términos que se describen a continuación:

        \begin{itemize}

          \item \textbf{Tiempo}\emph{(time)}: En la disciplina de \emph{Computación en Tiempo Real} el tiempo de ejecucción de una determinada tarea es especialmente crucial para garantizar el correcto desarrollo del cómputo, debido a que se asume un plazo de ejecucción permitido, a partir del cual la solución del problema deja de tener un valor óptimo. Shin y Ramanathan\cite{259423} diferencian entre tres categorías dentro de dicha restricción, a las cuales denominan \emph{hard}, \emph{firm} y \emph{soft}, dependiendo del grado de relajación de la misma.

          \item \textbf{Confiabilidad}\emph{(correctness)}: Otro de los puntos cruciales en un sistema de \emph{Cómputación en Tiempo Real} es la determinación de una unidad de medida o indicador acerca de las garantias de una determinada solución algorítmica para cumplir lo que promete de manera correcta en el tiempo esperado.

          \item \textbf{Entorno}\emph{(environment)}: El último factor que indican Shin y Ramanathan\cite{259423} para describir un sistema de \emph{Computación en Tiempo Real} es el entorno del mismo, debido a que este condiciona el conjunto de tareas y la periodicidad en que se deben llevar a cabo. Debido a esta razón, realizan una diferenciación entre
          \begin{enumerate*}[label=\itshape\alph*\upshape)]
  					\item tareas periódicas \emph{periodic tasks} las cuales se realizan secuencialmente a partir de la finalización de una ventana de tiempo, y
  					\item tareas no periódicas \emph{periodic tasks} que se llevan a cabo debido al suceso de un determinado evento externo.
  				\end{enumerate*}

        \end{itemize}


      \subsection{Problemas Dinámicos}
      \label{sec:dynamic_problems}

        \paragraph{}
        Una vez completada la descripción acerca de lo que se puede definir como \emph{Computación en Tiempo Real}, conviene realizar una descripción desde el punto de vista de la \emph{teoría de complejidad computacional}. Para definir este tipo de problemas, se utiliza el término \emph{problemas dinámicos}, los cuales consisten en aquellos en los cuales es necesario recalcular su solución conforme el tiempo avanza debido a variaciones en los parámetros de entrada del problema (Nótese que dicho término no debe confundirse con la estrategia de \emph{programación dinámica} para el diseño de algoritmos). Existen distintas vertientes dependiendo del punto de vista desde el que se estudien, tanto de la naturaleza del problema (soluciones dependientes temporalmente unas de otras o soluciones aisladas) como de los parámetros de entrada (entrada completa en cada nueva ejecución o variación respecto de la anterior). Los \emph{Algoritmos para Streaming} están diseñados para resolver \emph{problemas dinámicos}, por lo que en la sección \ref{sec:streaming_model}, se describe en profundidad el modelo en que se enmarcan.

        \paragraph{}
        A continuación se indican los principales indicadores utilizados para describir la complejidad de una determinada solución algorítmica destinada a resolver un problema de dicha naturaleza:

        \begin{itemize}
          \item Espacio: Cantidad de espacio utilizado en memoria durante la ejecución del algoritmo.
          \item Inicialización: Tiempo necesario para la inicialización del algoritmo.
          \item Procesado: Tiempo necesario para procesar una determinada entrada.
          \item Pregunta[TODO Buscar mejor palabra]: Tiempo necesario para procesar la solución a partir de los datos de entrada procesados hasta el momento.
        \end{itemize}


      \subsection{Algoritmos Online vs Algoritmos Offline}

        \paragraph{}
        Una vez descrita la problemática de \emph{Computación en Tiempo Real} en la sección \ref{sec:realtime_computing} y la categoría de \emph{Problemas Dinámicos} en la sección \ref{sec:dynamic_problems}, en esta sección se pretende ilustrar la diferencia entre los \emph{Algoritmos Online} y los \emph{Algoritmos Offline}. Para ello, se ha seguido la diferenciación propuesta por Karp \cite{Karp:1992:OAV:645569.659725}, en la cual se plantea el problema de la siguiente manera (Se utilizará la misma notación que sigue Muthukrishnan\cite{Muthukrishnan:2005:DSA:1166409.1166410} para tratar mantener la consistencia durante todo el documento): Sea $A$ el conjunto de datos o eventos de entrada, siendo cada $A[i]$ el elemento \emph{i-ésimo} del conjunto. En el caso de los \emph{Algoritmos Online} supondremos que es el elemento recibido en el instante \emph{i}. A continuación se muestran las características de cada subgrupo:

        \begin{itemize}
          \item \textbf{Algoritmos Offline}: Esta categoría contiene todos los algoritmos que realizan el cómputo suponiendo el acceso a cualquier elemento del conjunto de datos $A$ durante cualquier momento de su ejecución. Además, en esta categoría se impore la restricción de que el $A$ debe ser invariante respecto del tiempo, lo que impone que para la adaptación del resultado a cambios, este tenga que realizar una nueva ejecución desde su estado inicial. Nótese que por tanto, dentro de este grupo se engloba la mayoría de algoritmos utilizados comunmente.

          \item \textbf{Algoritmos Online}: Son aquellos que calculan el resultado a partir de una secuencia de sucesos $A[i]$, los cuales generan un resultado dependiente del valor, y posiblemente de los sucedidos posteriormente. A partir de dicha estrategia, se añade una componente dinámica, la cual permite que tamaño del conjunto de datos de entrada $A$ no tenga impuesta una restricción acerca de su longitud. Por contra, en este modelo no se permite conocer el cuceso $A[i+1]$ en el momento $i$.  Esto encaja perfectamente en el modelo que se describirá en la sección \ref{sec:streaming_model}.

        \end{itemize}

        \paragraph{}
        Según la diferenciación que se acaba de describir, estas dos estrategias de diseño de algoritmos encajan en disciplinas distintas de diseño de algoritmos, teniendo una gran ventaja a nivel de eficiencia en el caso estático los \emph{Algoritmos Offline}, pero quedando prácticamente inutilizables cuando la computación es en tiempo real, donde es mucho más apropiado el uso de estrategias de diseño de \emph{Algoritmos Online}.

        \paragraph{}
        Como medida de eficiencia para los \emph{Algoritmos Online}, Karp \cite{Karp:1992:OAV:645569.659725} proponen el indicador de \textbf{Ratio Competitivo}, el cual se define como la cota inferior del coste de cualquier nueva entrada con respecto de la que tiene menor coste. Sin embargo, dicha medida de eficiencia no es comúnmente utilizada en el caso de los \emph{Algoritmos para Streaming} por la componente estocástica de los mismos, para los cuales son más apropiadas medidas probabilistas. A continuación se describen las ventajas de estos respecto de su vertiente estática.

      \subsection{Algoritmos Online Probabilistas vs Deterministas}

        \paragraph{}
        [TODO]

    \section{Modelo en Streaming}
    \label{sec:streaming_model}

      \paragraph{}
      [TODO]

      \subsection{Modelo de Serie Temporal}
        \paragraph{}
        [TODO]

      \subsection{Modelo de Caja Registradora}
        \paragraph{}
        [TODO]

      \subsection{Modelo de Molinete}
        \paragraph{}
        [TODO]

    \section{Estructura básica}
    \label{sec:streaming_structure}

      \paragraph{}
      [TODO]

    \section{Medidas de Análisis}
    \label{sec:streaming_analysis}

\end{document}
